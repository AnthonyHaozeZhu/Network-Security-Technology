% -*- coding: utf-8 -*-
%-------------------------designed by zcf--------------
\documentclass[UTF8,a4paper,10pt]{ctexart}
\usepackage[left=3.17cm, right=3.17cm, top=2.74cm, bottom=2.74cm]{geometry}
\usepackage{amsmath}
\usepackage{graphicx,subfig}
\usepackage{float}
\usepackage{cite}
\usepackage{caption}
\usepackage{enumerate}
\usepackage{booktabs} %表格
\usepackage{multirow}
\newcommand{\tabincell}[2]{\begin{tabular}{@{}#1@{}}#2\end{tabular}}  %表格强制换行
%-------------------------字体设置--------------
\usepackage{times} 
\newcommand{\yihao}{\fontsize{26pt}{36pt}\selectfont}           % 一号, 1.4 倍行距
\newcommand{\erhao}{\fontsize{22pt}{28pt}\selectfont}          % 二号, 1.25倍行距
\newcommand{\xiaoer}{\fontsize{18pt}{18pt}\selectfont}          % 小二, 单倍行距
\newcommand{\sanhao}{\fontsize{16pt}{24pt}\selectfont}  %三号字
\newcommand{\xiaosan}{\fontsize{15pt}{22pt}\selectfont}        % 小三, 1.5倍行距
\newcommand{\sihao}{\fontsize{14pt}{21pt}\selectfont}            % 四号, 1.5 倍行距
\newcommand{\banxiaosi}{\fontsize{13pt}{19.5pt}\selectfont}    % 半小四, 1.5倍行距
\newcommand{\xiaosi}{\fontsize{12pt}{18pt}\selectfont}            % 小四, 1.5倍行距
\newcommand{\dawuhao}{\fontsize{11pt}{11pt}\selectfont}       % 大五号, 单倍行距
\newcommand{\wuhao}{\fontsize{10.5pt}{15.75pt}\selectfont}    % 五号, 单倍行距
%-------------------------章节名----------------
\usepackage{ctexcap} 
\CTEXsetup[name={,、},number={ \chinese{section}}]{section}
\CTEXsetup[name={(,)},number={\chinese{subsection}}]{subsection}
\CTEXsetup[name={,.},number={\arabic{subsubsection}}]{subsubsection}
%-------------------------页眉页脚--------------
\usepackage{fancyhdr}
\pagestyle{fancy}
\lhead{\kaishu \leftmark}
% \chead{}
\rhead{\kaishu 网络安全技术实验报告}%加粗\bfseries 
\lfoot{}
\cfoot{\thepage}
\rfoot{}
\renewcommand{\headrulewidth}{0.1pt}  
\renewcommand{\footrulewidth}{0pt}%去掉横线
\newcommand{\HRule}{\rule{\linewidth}{0.5mm}}%标题横线
\newcommand{\HRulegrossa}{\rule{\linewidth}{1.2mm}}
%-----------------------伪代码------------------
\usepackage{algorithm}  
\usepackage{algorithmicx}  
\usepackage{algpseudocode}  
\floatname{algorithm}{Algorithm}  
\renewcommand{\algorithmicrequire}{\textbf{Input:}}  
\renewcommand{\algorithmicensure}{\textbf{Output:}} 
\usepackage{lipsum}  
\makeatletter
\newenvironment{breakablealgorithm}
  {% \begin{breakablealgorithm}
  \begin{center}
     \refstepcounter{algorithm}% New algorithm
     \hrule height.8pt depth0pt \kern2pt% \@fs@pre for \@fs@ruled
     \renewcommand{\caption}[2][\relax]{% Make a new \caption
      {\raggedright\textbf{\ALG@name~\thealgorithm} ##2\par}%
      \ifx\relax##1\relax % #1 is \relax
         \addcontentsline{loa}{algorithm}{\protect\numberline{\thealgorithm}##2}%
      \else % #1 is not \relax
         \addcontentsline{loa}{algorithm}{\protect\numberline{\thealgorithm}##1}%
      \fi
      \kern2pt\hrule\kern2pt
     }
  }{% \end{breakablealgorithm}
     \kern2pt\hrule\relax% \@fs@post for \@fs@ruled
  \end{center}
  }
\makeatother
%------------------------代码-------------------
\usepackage{xcolor} 
\usepackage{listings} 
\usepackage{fontspec}
\newfontfamily\menlo{Menlo}
\setmonofont[Mapping={}]{Monaco} 
\definecolor{mygreen}{rgb}{0,0.6,0}
\definecolor{mygray}{rgb}{0.5,0.5,0.5}
\definecolor{mymauve}{rgb}{0.58,0,0.82}
\lstset{ %
backgroundcolor=\color{white},   % choose the background color
basicstyle=\footnotesize\ttfamily,        % size of fonts used for the code
columns=fullflexible,
breaklines=true,                 % automatic line breaking only at whitespace
captionpos=b,                    % sets the caption-position to bottom
tabsize=4,
commentstyle=\color{mygreen},    % comment style
escapeinside={\%*}{*)},          % if you want to add LaTeX within your code
keywordstyle=\color{blue},       % keyword style
stringstyle=\color{mymauve}\ttfamily,     % string literal style
frame=single,
rulesepcolor=\color{red!20!green!20!blue!20},
numbers=left,
 numberstyle=\tiny\menlo
% identifierstyle=\color{red},
% language=c++,
}
%------------超链接----------
\usepackage[colorlinks,linkcolor=black,anchorcolor=blue]{hyperref}
%------------------------TODO-------------------
\usepackage{enumitem,amssymb}
\newlist{todolist}{itemize}{2}
\setlist[todolist]{label=$\square$}
% for check symbol 
\usepackage{pifont}
\newcommand{\cmark}{\ding{51}}%
\newcommand{\xmark}{\ding{55}}%
\newcommand{\done}{\rlap{$\square$}{\raisebox{2pt}{\large\hspace{1pt}\cmark}}\hspace{-2.5pt}}
\newcommand{\wontfix}{\rlap{$\square$}{\large\hspace{1pt}\xmark}}
%------------------------水印-------------------
\usepackage{tikz}
\usepackage{xcolor}
\usepackage{eso-pic}

\newcommand{\watermark}[3]{\AddToShipoutPictureBG{
\parbox[b][\paperheight]{\paperwidth}{
\vfill%
\centering%
\tikz[remember picture, overlay]%
  \node [rotate = #1, scale = #2] at (current page.center)%
    {\textcolor{gray!80!cyan!30!magenta!30}{#3}};
\vfill}}}



%———————————————————————————————————————————正文———————————————————————————————————————————————
%----------------------------------------------
\begin{document}
\begin{titlepage}
    \begin{center}
    \includegraphics[width=0.8\textwidth]{NKU.png}\\[1cm]    
    \textsc{\Huge \kaishu{\textbf{南\ \ \ \ \ \ 开\ \ \ \ \ \ 大\ \ \ \ \ \ 学}} }\\[0.9cm]
    \textsc{\huge \kaishu{\textbf{计\ \ 算\ \ 机\ \ 学\ \ 院}}}\\[0.5cm]
    \textsc{\Large \textbf{网络安全技术作业报告}}\\[0.8cm]
    \HRule \\[0.9cm]
    { \LARGE \bfseries 端口扫描器的设计与实现}\\[0.4cm]
    \HRule \\[2.0cm]
    \centering
    \textsc{\LARGE \kaishu{朱浩泽\ 1911530}}\\[0.5cm]
    \textsc{\LARGE \kaishu{年级\ :\ 2019级}}\\[0.5cm]
    \textsc{\LARGE \kaishu{专业\ :\ 计算机科学与技术}}\\[0.5cm]
    \textsc{\LARGE \kaishu{班级\ :\ 计算机科学与技术2班}}\\[0.5cm]
    \vfill
    {\Large \today}
    \end{center}
\end{titlepage}
% -------------摘------要--------------
\newpage
\thispagestyle{empty}
% ----------------------------------------------------------------
\tableofcontents
% ----------------------------------------------------------------
\newpage
\watermark{60}{10}{NKU}
\setcounter{page}{1}
\section{实验介绍}
端口扫描器是一种重要的网络安全检测工具。通过端口扫描,不仅可以发现目标主机的 开放端口和操作系统的类型,还可以查找系统的安全漏洞,获得弱口令等相关信息。因此, 端口扫描技术是网络安全的基本技术之一,对于维护系统的安全性有着十分重要的意义。
\subsection{实验目的}
\begin{itemize}
    \item 掌握端口扫描器的基本设计方法。
    \item 理解 ping 程序,TCP connect 扫描,TCP SYN 扫描,TCP FIN 扫描以及 UDP 扫描的 工作原理。
    \item 熟练掌握 Linux 环境下的套接字编程技术。
    \item 掌握 Linux 环境下多线程编程的基本方法
\end{itemize}

\subsection{实验流程}
\begin{itemize}
    \item 编写端口扫描程序,提供 TCP connect 扫描
    \item 编写端口扫描程序,提供TCP SYN 扫描
    \item 编写端口扫描程序,提供TCP FIN 扫描
    \item 编写端口扫描程序,提供 UDP 扫描
    \item 设计并实现 ping 程序,探测目标主机是否可达。
\end{itemize}


\section{实验步骤}
\subsection{实验环境}
Ubuntu Server 20.04 LTS 64bit(腾讯云服务器), C++11, Cmake
\subsection{核心代码实现}
\subsubsection{代码框架}
\begin{lstlisting}
├── include
│   └── header.h
└── src
    ├── TCPConnectScan.cpp
    ├── TCPFINScan.cpp
    ├── TCPSYNScan.cpp
    ├── UDPScan.cpp
    └── main.cpp
\end{lstlisting}
header.h中包含了端口扫描器的基本结构体和函数声明。TCPConnectScan.cpp
、TCPFINScan.cpp、TCPSYNScan.cpp、UDPScan.cpp和main.cpp中分别实现了 TCP connect 扫描、TCP FIN 扫描、TCP SYN 扫描、UDP 扫描以及 ping 程序的实现。

\subsubsection{IP头、TCP头和TCP伪头以及一些工具函数}
TCP头,用于发送TCP报文
\begin{lstlisting}
struct TCPHeader {
      uint16_t srcPort;
      uint16_t dstPort;
      uint32_t seq;
      uint32_t ack;
      uint8_t null1 : 4;
      uint8_t length : 4;
      uint8_t FIN : 1;
      uint8_t SYN : 1;
      uint8_t RST : 1;
      uint8_t PSH : 1;
      uint8_t ACK : 1;
      uint8_t URG : 1;
      uint8_t null2 : 2;
      uint16_t windowSize;
      uint16_t checkSum;
      uint16_t ptr;
  };
\end{lstlisting}

TCP伪头,用于计算TCP头的校验和
\begin{lstlisting}[language=C++]
struct pseudohdr   
{  
   unsigned int saddr; 
   unsigned int daddr; 
   char useless; 
   unsigned char protocol; 
   unsigned short length; 
};
\end{lstlisting}

IP头,用于发送IP报文
\begin{lstlisting}[language = C++]
struct IPHeader {
    unsigned char headerLen : 4;
    unsigned char version : 4;
    unsigned char tos;
    unsigned short length;
    unsigned short ident;
    unsigned short fragFlags;
    unsigned char ttl;
    unsigned char protocol;
    unsigned short checksum;
    unsigned int srcIP;
    unsigned int dstIP;
    IPHeader(unsigned int src, unsigned int dst, int protocol) {
        version = 4;
        headerLen = 5;
        srcIP = src;
        dstIP = dst;
        ttl = (char)128;
        this -> protocol = protocol;
        if (protocol == IPPROTO_TCP) {
            length = htons(20 + 20);
        } 
		else if (protocol == IPPROTO_UDP) {
            length = htons(20 + 8);
        }
    }
};
\end{lstlisting}

校验和计算函数
\begin{lstlisting}[language=C++]
static inline unsigned short in_cksum(unsigned short *ptr, int nbytes) 
{ 
	register long sum; 
	u_short oddbyte; 
	register u_short answer; 

	sum = 0; 
	while(nbytes > 1) 
	{ 
		sum += *ptr++; 
		nbytes -= 2; 
	} 

	if(nbytes == 1) 
	{ 
		oddbyte = 0; 
		*((u_char *) &oddbyte) = *(u_char *)ptr; 
		sum += oddbyte; 
	} 

	sum = (sum >> 16) + (sum & 0xffff); 
	sum += (sum >> 16); 
	answer = ~sum; 

	return(answer); 
}
\end{lstlisting}

获取本地IP地址
\begin{lstlisting}[language = C++]
static inline unsigned int GetLocalHostIP(void) 
{ 
	FILE *fd; 
	char buf[20] = {0x00}; 

	fd = popen("/sbin/ifconfig | grep inet | grep -v 127 | awk '{print $2}' | cut -d \":\" -f 2", "r"); 
	if(fd == NULL)
	{ 
		fprintf(stderr, "cannot get source ip -> use the -f option\n"); 
		exit(-1); 
	} 
	fscanf(fd, "%20s", buf); 
	return(inet_addr(buf)); 
} 
\end{lstlisting}

\subsubsection{ICMP 探测指定主机}
该程序用于测量本地主机与目标主机之间的网络通信情况,用ping函数实现。具体实现为首先我们需要建立一个套接字用来通信,并设置我们需要发送的IP包
\begin{lstlisting}[language = C++]
bool Ping(std::string HostIP, unsigned LocalHostIP) {
    struct iphdr *ip; 
    struct icmphdr *icmp;
    unsigned short LocalPort = 8888;

    int PingSock = socket(AF_INET, SOCK_RAW, IPPROTO_ICMP);

    if(PingSock < 0) {
        std::cout << "socket error" << std::endl;
        return false;
    }
    
    int on = 1;
    int ret = setsockopt(PingSock, 0, IP_HDRINCL, &on, sizeof(on));

    if(ret < 0) {
        std::cout << "setsockopt error" << std::endl;
        return false;
    }
\end{lstlisting}
然后我们创建 ICMP 请求数据包,并对ip头和icmp头进行填充,为了保证对面成功接受并进行应答
\begin{lstlisting}[language = C++]
   int SendBufSize = sizeof(struct iphdr) + sizeof(struct icmphdr) + sizeof(struct timeval);
    char *SendBuf = (char*)malloc(SendBufSize);
    memset(SendBuf, 0, sizeof(SendBuf));

    ip = (struct iphdr*)SendBuf;
    ip -> ihl = 5;
    ip -> version = 4;
    ip -> tos = 0;
    ip -> tot_len = htons(SendBufSize);
    ip -> id = rand();
    ip -> ttl = 64;
    ip -> frag_off = 0x40;
    ip -> protocol = IPPROTO_ICMP;
    ip -> check = 0;
    ip -> saddr = LocalHostIP;
    ip -> daddr = inet_addr(&HostIP[0]);

    //填充icmp头
    icmp = (struct icmphdr*)(ip + 1);
    icmp->type = ICMP_ECHO;
    icmp->code = 0;
    icmp->un.echo.id = htons(LocalPort);
    icmp->un.echo.sequence = 0;

    struct timeval *tp = (struct timeval*) &SendBuf[28];
    gettimeofday(tp, NULL);
    icmp -> checksum = in_cksum((u_short *)icmp, sizeof(struct icmphdr) + sizeof(struct timeval));
\end{lstlisting}
然后我们设置套接字的发送地址,即我们需要扫描的目标地址,并向目标地址发送我们的icmp的数据包。
\begin{lstlisting}[language = C++]
   //设置套接字的发送地址
    struct sockaddr_in PingHostAddr;
    PingHostAddr.sin_family = AF_INET;
    PingHostAddr.sin_addr.s_addr = inet_addr(&HostIP[0]);
    int Addrlen = sizeof(struct sockaddr_in);

    //发送ICMP请求
    ret = sendto(PingSock, SendBuf, SendBufSize, 0, (struct sockaddr*) &PingHostAddr, sizeof(PingHostAddr));
    if(ret < 0) {
        std::cout << "sendto error" << std::endl;
        return false;
    }

    if(fcntl(PingSock, F_SETFL, O_NONBLOCK) == -1) {
        perror("fcntl error");
        return false;
    }
\end{lstlisting}
然后循环接受icmp响应,具体为首先获得循环起始时间,然后创建一个ICMP接受数据包,进入循环,如果接收到一个数据包则对其进行解析,获得响应数据包的IP头的原地址、目的地址,然后判断该数据包的源地址是否等于被测主机的IP地址和目的地址是否相等ICMP 头的 type 字段是否为 ICMP\_ECHOREPLY,如果等待时间超过三秒则是失败。
\begin{lstlisting}[language = C++]
   struct timeval TpStart, TpEnd;
    bool flags;
    //循环等待接收ICMP响应
    gettimeofday(&TpStart, NULL); //获得循环起始时刻
    flags = false;

    char RecvBuf[1024];
    struct sockaddr_in FromAddr;
    struct icmp* Recvicmp;
    struct ip* Recvip;
    std::string SrcIP, DstIP, LocalIP;
    struct in_addr in_LocalhostIP;

    do {
        //接收ICMP响应
        ret = recvfrom(PingSock, RecvBuf, 1024, 0, (struct sockaddr*) &FromAddr,
        (socklen_t*) &Addrlen);
        if (ret > 0) //如果接收到一个数据包,对其进行解析
        {
            Recvip = (struct ip*) RecvBuf;
            Recvicmp = (struct icmp*) (RecvBuf + (Recvip -> ip_hl * 4));
            SrcIP = inet_ntoa(Recvip -> ip_src); //获得响应数据包IP头的源地址
            DstIP = inet_ntoa(Recvip -> ip_dst); //获得响应数据包IP头的目的地址
            in_LocalhostIP.s_addr = LocalHostIP;
            LocalIP = inet_ntoa(in_LocalhostIP); //获得本机IP地址
            //判断该数据包的源地址是否等于被测主机的IP地址,目的地址是否等于
            //本机IP地址,ICMP头的type字段是否为ICMP_ECHOREPLY
            if (SrcIP == HostIP && DstIP == LocalIP &&
            Recvicmp->icmp_type == ICMP_ECHOREPLY) { 
                /*ping成功,退出循环*/
                std::cout << "Ping Host " << HostIP << " Successfully !" << std::endl;
				flags =true;
				break;
            }
        }
        //获得当前时刻,判断等待相应时间是否超过3秒,若是,则退出等待。
        gettimeofday(&TpEnd, NULL);
        float TimeUse = (1000000 * (TpEnd.tv_sec - TpStart.tv_sec) + (TpEnd.tv_usec - TpStart.tv_usec)) / 1000000.0;
        if(TimeUse < 3) {
            continue; 
        }
        else {
            flags = false;
            break;
        }
    } while(true);
    return flags;
}
\end{lstlisting}

\subsubsection{TCP connect 扫描}
这一部分我们使用的数据结构如下:
\begin{lstlisting}[language = C++]
struct TCPConHostThrParam
{
	std::string HostIP;
	unsigned HostPort;
};

struct TCPConThrParam
{
	std::string HostIP;
	unsigned BeginPort;
	unsigned EndPort;
};
\end{lstlisting}
我们这一部分的主要功能是利用TCP扫描确定目的主机的某一TCP端口是否开启该,具体来收就是尝试连接被测主机的指定端口,若连接成功,则表示端口开启;否则,表示端口关闭。为了提高效率,我们使用多线程进行扫描,每个进程扫描一个端口。首先我们先创建两个线程锁,为了让我们扫描端口并行化扫描。使用变量 TCPConThrdNum来记录已经创建的子线程数。
\begin{lstlisting}[language = C++]
int TCPConThrdNum;
pthread_mutex_t TCPConPrintlocker = PTHREAD_MUTEX_INITIALIZER;
pthread_mutex_t TCPConScanlocker = PTHREAD_MUTEX_INITIALIZER;
\end{lstlisting}
然后我们编写void* Thread\_TCPconnectHost(void* param)函数,该该函数的主要功能是连接目标主机指定端口的工作。首先,我们获得目标主机的IP地址和扫描端口号,然后创建流套接字,进入连接区,加锁防止多个线程同时打印字符出现乱码。
\begin{lstlisting}[language = C++]
void* Thread_TCPconnectHost(void* param) {
   /*变量定义*/
   //获得目标主机的IP地址和扫描端口号
   struct TCPConHostThrParam *p = (struct TCPConHostThrParam*) param;
   std::string HostIP = p -> HostIP;
   unsigned HostPort = p -> HostPort;
   //创建流套接字
   int ConSock = socket(AF_INET,SOCK_STREAM,0);
   if(ConSock < 0) {
      pthread_mutex_lock(&TCPConPrintlocker);

   }
\end{lstlisting}
然后设置连接主机,利用connect函数连接目标主机,加锁防止多个线程同时打印出现输出乱码,若连接成功,则表示端口开启;否则,表示端口关闭。
\begin{lstlisting}[language = C++]
   //设置连接主机地址
    struct sockaddr_in HostAddr;
    memset(&HostAddr, 0, sizeof(HostAddr));
    HostAddr.sin_family = AF_INET;
    HostAddr.sin_addr.s_addr = inet_addr(&HostIP[0]);
    HostAddr.sin_port = htons(HostPort);
    //connect目标主机
    int ret = connect(ConSock, (struct sockaddr*) &HostAddr, sizeof(HostAddr));
    if(ret < 0) {
        pthread_mutex_lock(&TCPConPrintlocker);
        std::cout << "TCP connect scan: " << HostIP << ":" << HostPort << " is closed" << std::endl;
        pthread_mutex_unlock(&TCPConPrintlocker);
    } else {
        pthread_mutex_lock(&TCPConPrintlocker);
        std::cout << "TCP connect scan: " << HostIP << ":" << HostPort << " is open" << std::endl;
        pthread_mutex_unlock(&TCPConPrintlocker);
    }
\end{lstlisting}
然后我们关闭套接字,释放线程锁,线程数量减一。
\begin{lstlisting}[language = C++]
   delete p;
   close(ConSock); //关闭套接字
   //子线程数减1
   pthread_mutex_lock(&TCPConScanlocker);
   TCPConThrdNum--;
   pthread_mutex_unlock(&TCPConScanlocker);
} // TCP connect 扫描
\end{lstlisting}
然后,我们编写void* Thread\_TCPconnectHost(void* param)函数,该函数用于遍历目标主机的端口,是该功能的主线程函数。首先我们获得扫描的目标主机IP、启始端口、终止端口,然后我们将线程数设置为0。
\begin{lstlisting}[language = C++]
void* Thread_TCPconnectScan(void* param)
{
    /*变量定义*/
    //获得扫描的目标主机IP,启始端口,终止端口
    struct TCPConThrParam *p = (struct TCPConThrParam*) param;
    std::string HostIP = p -> HostIP;
    unsigned BeginPort = p -> BeginPort;
    unsigned EndPort = p->EndPort;
    TCPConThrdNum = 0; //将线程数设为0
    //开始从起始端口到终止端口循环扫描目标主机的端口
\end{lstlisting}
接下来,我们开始从起始端口到终止端口循环扫描目标主机的端口。首先我们在循环中设置子线程参数,然后将子线程设为分离状态,创建connect目标主机指定的端口和一个独立的子线程进行绑定,并将子线程数加1。每一寸循环都会判断子线程的数量,如果如果子线程数大于100,则暂时休眠。
\begin{lstlisting}[language = C++]
   pthread_t subThreadID;
	pthread_attr_t attr;
    for (unsigned TempPort = BeginPort; TempPort <= EndPort; TempPort++)
    {
        //设置子线程参数
        TCPConHostThrParam *pConHostParam = new TCPConHostThrParam;
        pConHostParam->HostIP = HostIP;
        pConHostParam->HostPort = TempPort;
        //将子线程设为分离状态
        pthread_attr_init(&attr);
        pthread_attr_setdetachstate(&attr,PTHREAD_CREATE_DETACHED);
        //创建connect目标主机指定的端口子线程
        int ret = pthread_create(&subThreadID, &attr, Thread_TCPconnectHost, pConHostParam);
        if(ret == -1) {
            std::cout << "Create TCP connect scan thread error!" << std::endl;
        }
        //线程数加1
        pthread_mutex_lock(&TCPConScanlocker);
        TCPConThrdNum++;
        pthread_mutex_unlock(&TCPConScanlocker);
        //如果子线程数大于100,暂时休眠
        while (TCPConThrdNum>100) {
            sleep(3); 
        }
    } 
\end{lstlisting}
最后,我们等待子线程数为0,返回。
\begin{lstlisting}[language = C++]
   while (TCPConThrdNum != 0) {
      sleep(1);
  }
  pthread_exit(NULL);
} 
\end{lstlisting}

\subsubsection{TCP SYN 扫描}
这一部分我们使用的数据结构如下:
\begin{lstlisting}[language=C++]
struct TCPSYNHostThrParam
{
	std::string HostIP;
	unsigned HostPort;
    unsigned LocalPort;
	unsigned LocalHostIP;
};

struct TCPSYNThrParam
{
	std::string HostIP;
	unsigned BeginPort;
	unsigned EndPort;
	unsigned LocalHostIP;
};
\end{lstlisting}
我们这一部分的主要功能是利用TCP SYN扫描确定目的主机的某一TCP端口是否开启该,具体来收就是尝试向被测主机的指定端口发送SYN报文,如果接收到ACK报文,则说明开启,否则则说明关闭。为了提高效率,我们使用多线程进行扫描,每个进程扫描一个端口。首先我们先创建两个线程锁,为了让我们扫描端口并行化扫描。使用变量TCPSynThrdNum来记录已经创建的子线程数。
\begin{lstlisting}[language = C++]
pthread_mutex_t TCPSynPrintlocker = PTHREAD_MUTEX_INITIALIZER;
pthread_mutex_t TCPSynScanlocker = PTHREAD_MUTEX_INITIALIZER;

int TCPSynThrdNum;
\end{lstlisting}
然后我们编写void* Thread\_TCPSYNHost(void* param)函数,该函数的主要功能是完成对目标主机指定端口的 TCP SYN 扫描。首先,我们获得目标主机的IP地址和扫描端口号
\begin{lstlisting}[language = C++]
void* Thread_TCPSYNHost(void* param) {
   /*变量定义*/
    //获得目标主机的IP地址和扫描端口号
    struct TCPSYNHostThrParam *p = (struct TCPSYNHostThrParam*) param;
    std::string HostIP = p -> HostIP;
    unsigned HostPort = p -> HostPort;
    unsigned LocalPort = p -> LocalPort;
	unsigned LocalHostIP = p -> LocalHostIP;

    struct sockaddr_in SYNScanHostAddr;
    memset(&SYNScanHostAddr, 0, sizeof(SYNScanHostAddr));
    SYNScanHostAddr.sin_family = AF_INET;
    SYNScanHostAddr.sin_addr.s_addr = inet_addr(HostIP.c_str());
    SYNScanHostAddr.sin_port = htons(HostPort);
\end{lstlisting}
然后我们创建套接字
\begin{lstlisting}[language = C++]
   int SynSock = socket(AF_INET, SOCK_RAW, IPPROTO_TCP);
   if(SynSock < 0) {
       pthread_mutex_lock(&TCPSynPrintlocker);
     std::cout << "Can't creat raw socket !" << std::endl;
     pthread_mutex_unlock(&TCPSynPrintlocker);
   }
   int flag = 1;
   if (setsockopt(SynSock, IPPROTO_IP, IP_HDRINCL, (void*)&flag, sizeof(int)) ==
       -1) {
       std::cout << "set IP_HDRINCL error.\n";
   }
\end{lstlisting}
填充TCP SYN数据包
\begin{lstlisting}[language = C++]
   char sendbuf[8192];
    char recvbuf[8192];
    struct pseudohdr *ptcph = (struct pseudohdr*) sendbuf;
    struct tcphdr *tcph = (struct tcphdr*)(sendbuf + sizeof(struct pseudohdr));
\end{lstlisting}
填充TCP伪头部,用于计算校验和
\begin{lstlisting}[language = C++]
   ptcph -> saddr = LocalHostIP;
   ptcph -> daddr = inet_addr(HostIP.c_str());
   in_addr src, dst;
   ptcph -> useless = 0;
   ptcph -> protocol = IPPROTO_TCP;
   ptcph -> length = htons(sizeof(struct tcphdr));
    
   src.s_addr = ptcph -> saddr;
   dst.s_addr = ptcph -> daddr;
\end{lstlisting}
填充TCP头
\begin{lstlisting}[language = C++]
   memset(tcph, 0, sizeof(struct tcphdr));
   // std::cout<<LocalPort<<" "<<HostPort<<std::endl;
   tcph->th_sport = htons(LocalPort);  
   tcph->th_dport = htons(HostPort); 
   tcph->th_seq = htonl(123456); 
   tcph->th_ack = 0; 
   tcph->th_x2 = 0; 
   tcph->th_off = 5; 
   tcph->th_flags = TH_SYN; 
   tcph->th_win = htons(65535); 
   tcph->th_sum = 0; 
   tcph->th_urp = 0; 
   tcph->th_sum = in_cksum((unsigned short*)ptcph, 20 + 12);
\end{lstlisting}
封装IP头
\begin{lstlisting}[language = C++]
   IPHeader IPheader(ptcph -> saddr, ptcph -> daddr, IPPROTO_TCP);
	char temp[sizeof(IPHeader) + sizeof(struct tcphdr)];

	memcpy((void*)temp, (void*)&IPheader, sizeof(IPheader));
	memcpy((void*)(temp+sizeof(IPheader)), (void*)tcph, sizeof(struct tcphdr));
\end{lstlisting}
发送TCP SYN数据包
\begin{lstlisting}[language = C++]
   int len = sendto(SynSock, temp, sizeof(IPHeader) + sizeof(struct tcphdr), 0, (struct sockaddr *)&SYNScanHostAddr, sizeof(SYNScanHostAddr));
    // std::cout << sizeof(IPHeader) <<" "<< sizeof(struct tcphdr)<<" "<<len << std::endl;
    if(len < 0) {
        pthread_mutex_lock(&TCPSynPrintlocker);
		std::cout << "Send TCP SYN Packet error !" << std::endl;
		pthread_mutex_unlock(&TCPSynPrintlocker);
    }
\end{lstlisting}
开始利用一个循环循环接受包到buffer中。
\begin{lstlisting}[language = C++]
   int count = 0;
   std::string SrcIP;
   struct ip *iph;
   flag = 1;
   sockaddr_in recvAddr;
   int addrLen = sizeof(recvAddr);
   do{
       len = recvfrom(SynSock, recvbuf, 8192, 0, (sockaddr*)&recvAddr,
                       (socklen_t*)&addrLen);
       if(len < 0) { 
           /*接收错误*/
           pthread_mutex_lock(&TCPSynPrintlocker);
           std::cout << "Read TCP SYN Packet error !" << std::endl;
           pthread_mutex_unlock(&TCPSynPrintlocker);
       }
\end{lstlisting}
解析IP头和TCP头,然后从TCP头和IP头中解析源地址、目的地址、源IP、目的IP
\begin{lstlisting}[language = C++]
       else {
           struct ip *iph = (struct ip *)recvbuf; 
           int i = iph -> ip_hl * 4; 
           tcph = (struct tcphdr *)(recvbuf + i); 

           std::string SrcIP = inet_ntoa(iph -> ip_src);
           std::string DstIP = inet_ntoa(iph -> ip_dst);
           struct in_addr in_LocalhostIP;
           in_LocalhostIP.s_addr = LocalHostIP;
           std::string LocalIP = inet_ntoa(in_LocalhostIP);

           unsigned SrcPort = ntohs(tcph -> th_sport);    
           unsigned DstPort = ntohs(tcph -> th_dport);  
\end{lstlisting}
判断响应数据包的源地址是否等于目标主机地址,目的地址是否等于本机;IP地址,源端口是否等于被扫描端口,目的端口是否等于本机端口号
\begin{lstlisting}[language = C++]
           // std::cout << "___________________________" << std::endl;

           // std::cout << HostIP << ' ' << SrcIP << std::endl;
           // std::cout << LocalIP << ' ' << DstIP << std::endl;
           // std::cout << SrcPort << ' ' << HostPort<< std::endl;
           // std::cout << DstPort << ' ' << LocalPort<< std::endl;
           if(HostIP == SrcIP && LocalIP == DstIP && SrcPort == HostPort && DstPort == LocalPort)
           {
\end{lstlisting}
判断数据包类型,给出动作响应。只让这个过程循环20次,如果没收到默认关闭。
\begin{lstlisting}[language = C++]
          
               // std::cout<<(int)(tcph->th_flags)<<std::endl;
               if(tcph->th_flags == 0x12) //判断是否为SYN|ACK数据包
               {
                   /*端口开启*/
                   flag = 0;
                   pthread_mutex_lock(&TCPSynPrintlocker);
                   std::cout << "Host: " << SrcIP << " Port: " << ntohs(tcph -> th_sport) << " open !" << std::endl;
                   pthread_mutex_unlock(&TCPSynPrintlocker);
               }
               if(tcph->th_flags == 0x14) //判断是否为RST数据包
               { 
                   /*端口关闭*/
                   flag = 0; 
                   pthread_mutex_lock(&TCPSynPrintlocker);
                   std::cout << " Port: " << ntohs(tcph -> th_sport) << " closed !" << std::endl;
                   pthread_mutex_unlock(&TCPSynPrintlocker); 
               }
           }
       }
   } while(count++ < 20 && flag);
\end{lstlisting}
最后,我们等待子线程数为0,返回。
\begin{lstlisting}[language = C++]
   //退出子线程
    if(flag){
        pthread_mutex_lock(&TCPSynPrintlocker);
        std::cout << "Host: " << SrcIP << " Port: " << HostPort << " closed !" << std::endl;
        pthread_mutex_unlock(&TCPSynPrintlocker);
    }
    delete p;
    close(SynSock);
    pthread_mutex_lock(&TCPSynScanlocker);
    TCPSynThrdNum--;
    pthread_mutex_unlock(&TCPSynScanlocker);
}
\end{lstlisting}
然后我们进行编写void* Thread\_TCPSynScan(void* param)函数,该函数的主要功能是调用 Thread\_TCPSYNHost 函数创建多个扫描子线程负责遍历目标主机的被测端口。首先我们获得目标主机的IP地址和扫描的起始端口号,终止端口号,以及本机的IP地址
\begin{lstlisting}[language = C++]
   void* Thread_TCPSynScan(void* param) {
    /*变量定义*/
    //获得目标主机的IP地址和扫描的起始端口号,终止端口号,以及本机的IP地址
    struct TCPSYNThrParam *p = (struct TCPSYNThrParam*)param;
    std::string HostIP = p -> HostIP;
    unsigned BeginPort = p-> BeginPort;
	unsigned EndPort = p-> EndPort;
	unsigned LocalHostIP = p -> LocalHostIP;
\end{lstlisting}
接下来,我们开始从起始端口到终止端口循环扫描目标主机的端口。首先我们在循环中设置子线程参数,然后将子线程设为分离状态,创建SYN目标主机指定的端口和一个独立的子线程进行绑定,并将子线程数加1。每一寸循环都会判断子线程的数量,如果如果子线程数大于100,则暂时休眠。
\begin{lstlisting}[language = C++]
   TCPSynThrdNum = 0;
    unsigned LocalPort = 1024;
    pthread_attr_t attr,lattr;
    pthread_t listenThreadID,subThreadID;
    for (unsigned TempPort = BeginPort; TempPort <= EndPort; TempPort++)
    {
        //设置子线程参数
        struct TCPSYNHostThrParam *pTCPSYNHostParam =
        new TCPSYNHostThrParam;
        pTCPSYNHostParam->HostIP = HostIP;
        pTCPSYNHostParam->HostPort = TempPort;
        pTCPSYNHostParam->LocalPort = TempPort + LocalPort;
        pTCPSYNHostParam->LocalHostIP = LocalHostIP;
        //将子线程设置为分离状态
        pthread_attr_init(&attr);
        pthread_attr_setdetachstate(&attr,PTHREAD_CREATE_DETACHED);
        //创建子线程
        int ret = pthread_create(&subThreadID, &attr, Thread_TCPSYNHost, pTCPSYNHostParam);
        if (ret==-1) 
		{
			std::cout << "Can't create the TCP SYN Scan Host thread !" << std::endl;
		}
        pthread_attr_destroy(&attr);
        //子线程数加1
        pthread_mutex_lock(&TCPSynScanlocker);
        TCPSynThrdNum++;
        pthread_mutex_unlock(&TCPSynScanlocker);
        //子线程数大于100,休眠
        while(TCPSynThrdNum > 100) { 
            sleep(3); 
        }
    }
\end{lstlisting}
最后,我们等待子线程数为0,返回。
\begin{lstlisting}[language = C++]
   while(TCPSynThrdNum != 0) { 
      sleep(1);
  }
  //返回主流程
  pthread_exit(NULL);
}  
\end{lstlisting}


\subsubsection{TCP FIN 扫描}
这一部分我们需要使用的数据结构如下:
\begin{lstlisting}[language = C++]
struct TCPFINHostThrParam
{
    std::string HostIP;
    unsigned HostPort;
    unsigned LocalPort;
    unsigned LocalHostIP;
};

struct TCPFINThrParam
{
	std::string HostIP;
	unsigned BeginPort;
	unsigned EndPort;
	unsigned LocalHostIP;
};
\end{lstlisting}
我们这一部分的主要功能是利用TCP FIN扫描确定目的主机的某一TCP端口是否开启,具体来说就尝试向被测主机发送FIN报文,如果接收到ACK报文,则说明开启,否则则说明关闭。未来提高效率,我们使用多线程进行扫描,每个进程扫描一个端口。首先我们先创建两个线程锁,为了让我们扫描端口并行化进行。使用变量TCPFinThrdNum来记录已经创建的子进程数
\begin{lstlisting}[language = C++]
int TCPFinThrdNum;
pthread_mutex_t TCPFinPrintlocker = PTHREAD_MUTEX_INITIALIZER;
pthread_mutex_t TCPFinScanlocker = PTHREAD_MUTEX_INITIALIZER;
\end{lstlisting}
然后我们编写void* Thread\_TCPFINHost(void* param)函数,该函数的主要目的是完成对目标主机制定端口的TCP FIN扫描。首先,我们获得目标主机的IP地址和扫描端口号
\begin{lstlisting}[language = C++]
void* Thread_TCPFINHost(void* param) {
    /*------------与 TCP SYN 扫描类似-----------------*/ 
    //填充 TCP FIN 数据包
    struct TCPFINHostThrParam *p = (struct TCPFINHostThrParam*)param;
    std::string HostIP = p -> HostIP;
	unsigned HostPort = p->HostPort;
	unsigned LocalPort = p->LocalPort;
	unsigned LocalHostIP = p->LocalHostIP;

    struct sockaddr_in FINScanHostAddr;
    memset(&FINScanHostAddr, 0, sizeof(FINScanHostAddr));
	FINScanHostAddr.sin_family = AF_INET;
	FINScanHostAddr.sin_addr.s_addr = inet_addr(&HostIP[0]);
	FINScanHostAddr.sin_port = htons(HostPort); 
\end{lstlisting}
然后我们创建两个套接字,一个用于发送FIN报文,一个用于接收回复
\begin{lstlisting}[language = C++]
    int FinSock=socket(PF_INET, SOCK_RAW, IPPROTO_TCP); 
	if (FinSock < 0) {
		pthread_mutex_lock(&TCPFinPrintlocker);
		std::cout << "Can't creat raw socket !" << std::endl;
		pthread_mutex_unlock(&TCPFinPrintlocker);
	}

    int FinRevSock = socket(PF_INET, SOCK_RAW, IPPROTO_TCP); 
	if (FinRevSock < 0) 
	{
		pthread_mutex_lock(&TCPFinPrintlocker);
		std::cout << "Can't creat raw socket !" << std::endl;
		pthread_mutex_unlock(&TCPFinPrintlocker);
	}
	int flag = 1;
    if (setsockopt(FinSock, IPPROTO_IP, IP_HDRINCL, (void*)&flag, sizeof(int)) ==
        -1) {
        std::cout << "set IP_HDRINCL error.\n";
    }
	if (setsockopt(FinRevSock, IPPROTO_IP, IP_HDRINCL, (void*)&flag, sizeof(int)) ==
        -1) {
        std::cout << "set IP_HDRINCL error.\n";
    }

\end{lstlisting}
填充TCP头
\begin{lstlisting}[language = C++]
    char sendbuf[8192];
    struct pseudohdr *ptcph = (struct pseudohdr*)sendbuf; 
    struct tcphdr *tcph = (struct tcphdr*)(sendbuf + sizeof(struct pseudohdr)); 

    
    
	ptcph->saddr = LocalHostIP; 
    ptcph->daddr = inet_addr(&HostIP[0]); 
    ptcph->useless = 0; 
    ptcph->protocol = IPPROTO_TCP; 
    ptcph->length = htons(sizeof(struct tcphdr));


    tcph->th_sport = htons(LocalPort);  
    tcph->th_dport = htons(HostPort); 
    tcph->th_seq = htonl(123456); 
    tcph->th_ack = 0; 
    tcph->th_x2 = 0; 
    tcph->th_off = 5; 
    tcph->th_flags = TH_FIN; 
    tcph->th_win = htons(65535); 
    tcph->th_sum = 0; 
    tcph->th_urp = 0; 
	tcph->th_sum = in_cksum((unsigned short*)ptcph, 20 + 12);
\end{lstlisting}
封装IP头
\begin{lstlisting}[language = C++]
    IPHeader IPheader(ptcph -> saddr, ptcph -> daddr, IPPROTO_TCP);
	char temp[sizeof(IPHeader) + sizeof(struct tcphdr)];

	memcpy((void*)temp, (void*)&IPheader, sizeof(IPheader));
	memcpy((void*)(temp+sizeof(IPheader)), (void*)tcph, sizeof(struct tcphdr));
\end{lstlisting}
发送TCP FIN数据包,并将另一个套接字设置为非阻塞模式进行接收
\begin{lstlisting}[language = C++]
    int len = sendto(FinSock, temp, sizeof(IPHeader) + sizeof(struct tcphdr), 0, (struct sockaddr *)&FINScanHostAddr, sizeof(FINScanHostAddr)); 
	if(len < 0) 
	{
		pthread_mutex_lock(&TCPFinPrintlocker);
		std::cout << "Send TCP FIN Packet error !" << std::endl;
		pthread_mutex_unlock(&TCPFinPrintlocker);		
	} 
    if(fcntl(FinRevSock, F_SETFL, O_NONBLOCK) == -1) 
	{
		pthread_mutex_lock(&TCPFinPrintlocker);
        std::cout << "Set socket in non-blocked model fail !" << std::endl;
		pthread_mutex_unlock(&TCPFinPrintlocker);
	}

    int FromAddrLen = sizeof(struct sockaddr_in);
\end{lstlisting}
然后开始利用一个循环将收到的数据包存入buffer中
\begin{lstlisting}[language = C++]
    struct timeval TpStart,TpEnd; 
    char recvbuf[8192];
    struct sockaddr_in FromAddr;
    std::string SrcIP, DstIP, LocalIP;
    gettimeofday(&TpStart, NULL); //获得开始接收时刻 
    struct in_addr in_LocalhostIP;
    do {
        //调用 recvfrom 函数接收数据包
        len = recvfrom(FinRevSock, recvbuf, sizeof(recvbuf), 0, (struct sockaddr*) &FromAddr, (socklen_t*) &FromAddrLen);
\end{lstlisting}
然后我们需要判断响应包的原地址是不是等于目的主机地址,并解析IP头和TCP头,然后从TCP头和IP头中解析源地址、目的地址、源IP、目的IP
\begin{lstlisting}[language=C++]
    if(len > 0)
    {
        std::string SrcIP = inet_ntoa(FromAddr.sin_addr);
        if(1)
        {      
            //响应数据包的源地址等于目标主机地址 
            struct ip *iph = (struct ip *)recvbuf; 
            int i = iph -> ip_hl * 4; 
            struct tcphdr *tcph = (struct tcphdr *)&recvbuf[i]; 

            SrcIP = inet_ntoa(iph->ip_src);       
            DstIP = inet_ntoa(iph->ip_dst);       
            
            
            in_LocalhostIP.s_addr = LocalHostIP;
            LocalIP = inet_ntoa(in_LocalhostIP);  

            unsigned SrcPort = ntohs(tcph->th_sport);  
            unsigned DstPort = ntohs(tcph->th_dport); 
\end{lstlisting}
判断响应数据包的源地址是否等于目标主机,目的地址是否等于本地主机;IP地址,源端口是否等于被扫描端口,目的端口是否等于本机号
\begin{lstlisting}[language = C++]
    // std::cout << "___________________________" << std::endl;

				// std::cout << HostIP << ' ' << SrcIP << std::endl;
				// std::cout << LocalIP << ' ' << DstIP << std::endl;
				// std::cout << SrcPort << ' ' << HostPort<< std::endl;
				// std::cout << DstPort << ' ' << LocalPort<< std::endl;
                //判断响应数据包的源地址是否等于目标主机地址,目的地址是否等于本机 IP 地址,源端口是否等于被扫描端口,目的端口是否等于本机端口号
                if(HostIP == SrcIP && LocalIP == DstIP && SrcPort == HostPort && DstPort == LocalPort)
				{
\end{lstlisting}
然后我们判断数据包类型,给出动作响应,
\begin{lstlisting}[language = C++]
    if (tcph->th_flags == 0x14) 
    {  
        pthread_mutex_lock(&TCPFinPrintlocker);
        std::cout << "Host: " << SrcIP << " Port: " << ntohs(tcph -> th_sport) << " closed !" << std::endl;
        pthread_mutex_unlock(&TCPFinPrintlocker);
    }
    break;				
\end{lstlisting}
只让这个过程重复5秒,如果5秒内没有响应则判定为没有响应,继续进行扫描
\begin{lstlisting}[language = C++]
    //判断等待响应数据包时间是否超过 3 秒
    gettimeofday(&TpEnd, NULL);
    float TimeUse = (1000000 * (TpEnd.tv_sec - TpStart.tv_sec) + (TpEnd.tv_usec - TpStart.tv_usec)) / 1000000.0;
    if(TimeUse < 5)
    {
        continue;
    }
    else
    {
        //超时,扫描端口开启 
        pthread_mutex_lock(&TCPFinPrintlocker);
        std::cout << "Host: " << HostIP << " Port: " << HostPort << " open !" << std::endl;
        pthread_mutex_unlock(&TCPFinPrintlocker);
        break;
    }
}
while(true);
\end{lstlisting}
最后我们退出子进程,并返回
\begin{lstlisting}[language = C++]
    delete p;
    close(FinSock);
	close(FinRevSock);

	pthread_mutex_lock(&TCPFinScanlocker);
    TCPFinThrdNum--;
	pthread_mutex_unlock(&TCPFinScanlocker);
}
\end{lstlisting}
然后我们编写void* Thread\_TCPFinScan(void* param)函数,该函数的主要功能是
调用Thread\_TCPFINHost函数创建多个扫描子线程负责遍历目标主机的被测端口。首先我们获取目标主机的IP地址和扫描的起始端口号,终止端口号,以及本机的IP地址
\begin{lstlisting}[language = C++]
void* Thread_TCPFinScan(void* param) {
	struct TCPFINThrParam *p = (struct TCPFINThrParam*)param;
	std::string HostIP = p->HostIP;
	unsigned BeginPort = p->BeginPort;
	unsigned EndPort = p->EndPort;
	unsigned LocalHostIP = p->LocalHostIP;

	TCPFinThrdNum = 0;
	unsigned LocalPort = 1024;
\end{lstlisting}
接下来,我们开始从起始端口到终止端口循环扫描目标主机的端口。首先我们在循环中设置子线程参数,然后将子线程设为分离状态。创建FIN目标主机指定的端口和一个独立的子线程进行绑定,并将子线程数加1。每一次循环都会判断子线程的数量,如果如果子线程数大于100,则暂时休眠。
\begin{lstlisting}[language = C++]
    TCPFinThrdNum = 0;
	unsigned LocalPort = 1024;

    pthread_attr_t attr, lattr;
    pthread_t listenThreadID, subThreadID;
	for (unsigned TempPort = BeginPort; TempPort <= EndPort;TempPort++) 
	{
        struct TCPFINHostThrParam *pTCPFINHostParam = new TCPFINHostThrParam;
        pTCPFINHostParam->HostIP = HostIP;
		pTCPFINHostParam->HostPort = TempPort;
        pTCPFINHostParam->LocalPort = TempPort + LocalPort;
		pTCPFINHostParam->LocalHostIP = LocalHostIP;


		pthread_attr_init(&attr);
        pthread_attr_setdetachstate(&attr,PTHREAD_CREATE_DETACHED);
                

		int ret = pthread_create(&subThreadID,&attr,Thread_TCPFINHost,pTCPFINHostParam);
		if (ret == -1) 
		{
			std::cout << "Can't create the TCP FIN Scan Host thread !" << std::endl;
		}

		pthread_attr_destroy(&attr);
		pthread_mutex_lock(&TCPFinScanlocker);
        TCPFinThrdNum++;
		pthread_mutex_unlock(&TCPFinScanlocker);

		while (TCPFinThrdNum>100)
		{
			sleep(3);
		}
	}
\end{lstlisting}
最后,我们等待子线程数为0,返回
\begin{lstlisting}[language = C++]
    while (TCPFinThrdNum != 0)
	{
		sleep(1);
	}

    std::cout << "TCP FIN scan thread exit !" << std::endl;
	pthread_exit(NULL);
}
\end{lstlisting}









































































\subsubsection{端口扫描器程序 Scaner}
首先程序判断是否需要输出帮助信息,若是,则输出端口扫描器程序的帮助信息,
然后退出;否则,继续执行下面的步骤。
\begin{lstlisting}[language = C++]
int main(int argc,char *argv[]) { 
   std::unordered_map<std::string, void(*)(int, char*[])> mapOp = {{"-h", print_h}, {"-c", print_c}, {"-s", print_s}, {"-u", print_u}, {"-f", print_f}};
   if (argc != 2) { 
      std::cout << "参数错误,argc = " << argc << std::endl;
      return -1;
   }
   std::string op = argv[1];


   if (mapOp.find(op) != mapOp.end()) {
      mapOp[op](argc, argv);
      return 0;
  }
  return 0;
}
\end{lstlisting}
打印帮助信息函数如下:
\begin{lstlisting}[language = C++]
void print_h(int argc, char *argv[]) {
    std::cout << "Scaner:usage:\n" << "\t" << "[-h] --help information " << std::endl;
    std::cout << "\t" << "[-c] --TCP connect scan" << std::endl;
    std::cout << "\t" << "[-s] --TCP syn scan" << std::endl;
    std::cout << "\t" << "[-f] --TCP fin scan" << std::endl;
    std::cout << "\t" << "[-u] --UDP scan" << std::endl;
}
\end{lstlisting}






































\section{实验结论}



\end{document}