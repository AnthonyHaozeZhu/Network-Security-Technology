% -*- coding: utf-8 -*-
%-------------------------designed by zcf--------------
\documentclass[UTF8,a4paper,10pt]{ctexart}
\usepackage[left=3.17cm, right=3.17cm, top=2.74cm, bottom=2.74cm]{geometry}
\usepackage{amsmath}
\usepackage{graphicx,subfig}
\usepackage{float}
\usepackage{cite}
\usepackage{caption}
\usepackage{enumerate}
\usepackage{booktabs} %表格
\usepackage{multirow}
\newcommand{\tabincell}[2]{\begin{tabular}{@{}#1@{}}#2\end{tabular}}  %表格强制换行
%-------------------------字体设置--------------
\usepackage{times} 
\newcommand{\yihao}{\fontsize{26pt}{36pt}\selectfont}           % 一号, 1.4 倍行距
\newcommand{\erhao}{\fontsize{22pt}{28pt}\selectfont}          % 二号, 1.25倍行距
\newcommand{\xiaoer}{\fontsize{18pt}{18pt}\selectfont}          % 小二, 单倍行距
\newcommand{\sanhao}{\fontsize{16pt}{24pt}\selectfont}  %三号字
\newcommand{\xiaosan}{\fontsize{15pt}{22pt}\selectfont}        % 小三, 1.5倍行距
\newcommand{\sihao}{\fontsize{14pt}{21pt}\selectfont}            % 四号, 1.5 倍行距
\newcommand{\banxiaosi}{\fontsize{13pt}{19.5pt}\selectfont}    % 半小四, 1.5倍行距
\newcommand{\xiaosi}{\fontsize{12pt}{18pt}\selectfont}            % 小四, 1.5倍行距
\newcommand{\dawuhao}{\fontsize{11pt}{11pt}\selectfont}       % 大五号, 单倍行距
\newcommand{\wuhao}{\fontsize{10.5pt}{15.75pt}\selectfont}    % 五号, 单倍行距
%-------------------------章节名----------------
\usepackage{ctexcap} 
\CTEXsetup[name={,、},number={ \chinese{section}}]{section}
\CTEXsetup[name={(,)},number={\chinese{subsection}}]{subsection}
\CTEXsetup[name={,.},number={\arabic{subsubsection}}]{subsubsection}
%-------------------------页眉页脚--------------
\usepackage{fancyhdr}
\pagestyle{fancy}
\lhead{\kaishu \leftmark}
% \chead{}
\rhead{\kaishu 网络安全技术实验报告}%加粗\bfseries 
\lfoot{}
\cfoot{\thepage}
\rfoot{}
\renewcommand{\headrulewidth}{0.1pt}  
\renewcommand{\footrulewidth}{0pt}%去掉横线
\newcommand{\HRule}{\rule{\linewidth}{0.5mm}}%标题横线
\newcommand{\HRulegrossa}{\rule{\linewidth}{1.2mm}}
%-----------------------伪代码------------------
\usepackage{algorithm}  
\usepackage{algorithmicx}  
\usepackage{algpseudocode}  
\floatname{algorithm}{Algorithm}  
\renewcommand{\algorithmicrequire}{\textbf{Input:}}  
\renewcommand{\algorithmicensure}{\textbf{Output:}} 
\usepackage{lipsum}  
\makeatletter
\newenvironment{breakablealgorithm}
  {% \begin{breakablealgorithm}
  \begin{center}
     \refstepcounter{algorithm}% New algorithm
     \hrule height.8pt depth0pt \kern2pt% \@fs@pre for \@fs@ruled
     \renewcommand{\caption}[2][\relax]{% Make a new \caption
      {\raggedright\textbf{\ALG@name~\thealgorithm} ##2\par}%
      \ifx\relax##1\relax % #1 is \relax
         \addcontentsline{loa}{algorithm}{\protect\numberline{\thealgorithm}##2}%
      \else % #1 is not \relax
         \addcontentsline{loa}{algorithm}{\protect\numberline{\thealgorithm}##1}%
      \fi
      \kern2pt\hrule\kern2pt
     }
  }{% \end{breakablealgorithm}
     \kern2pt\hrule\relax% \@fs@post for \@fs@ruled
  \end{center}
  }
\makeatother
%------------------------代码-------------------
\usepackage{xcolor} 
\usepackage{listings} 
\usepackage{fontspec}
\newfontfamily\menlo{Menlo}
\setmonofont[Mapping={}]{Monaco} 
\definecolor{mygreen}{rgb}{0,0.6,0}
\definecolor{mygray}{rgb}{0.5,0.5,0.5}
\definecolor{mymauve}{rgb}{0.58,0,0.82}
\lstset{ %
backgroundcolor=\color{white},   % choose the background color
basicstyle=\footnotesize\ttfamily,        % size of fonts used for the code
columns=fullflexible,
breaklines=true,                 % automatic line breaking only at whitespace
captionpos=b,                    % sets the caption-position to bottom
tabsize=4,
commentstyle=\color{mygreen},    % comment style
escapeinside={\%*}{*)},          % if you want to add LaTeX within your code
keywordstyle=\color{blue},       % keyword style
stringstyle=\color{mymauve}\ttfamily,     % string literal style
frame=single,
rulesepcolor=\color{red!20!green!20!blue!20},
numbers=left,
 numberstyle=\tiny\menlo
% identifierstyle=\color{red},
% language=c++,
}
%------------超链接----------
\usepackage[colorlinks,linkcolor=black,anchorcolor=blue]{hyperref}
%------------------------TODO-------------------
\usepackage{enumitem,amssymb}
\newlist{todolist}{itemize}{2}
\setlist[todolist]{label=$\square$}
% for check symbol 
\usepackage{pifont}
\newcommand{\cmark}{\ding{51}}%
\newcommand{\xmark}{\ding{55}}%
\newcommand{\done}{\rlap{$\square$}{\raisebox{2pt}{\large\hspace{1pt}\cmark}}\hspace{-2.5pt}}
\newcommand{\wontfix}{\rlap{$\square$}{\large\hspace{1pt}\xmark}}
%------------------------水印-------------------
\usepackage{tikz}
\usepackage{xcolor}
\usepackage{eso-pic}

\newcommand{\watermark}[3]{\AddToShipoutPictureBG{
\parbox[b][\paperheight]{\paperwidth}{
\vfill%
\centering%
\tikz[remember picture, overlay]%
  \node [rotate = #1, scale = #2] at (current page.center)%
    {\textcolor{gray!80!cyan!30!magenta!30}{#3}};
\vfill}}}



%———————————————————————————————————————————正文———————————————————————————————————————————————
%----------------------------------------------
\begin{document}
\begin{titlepage}
    \begin{center}
    \includegraphics[width=0.8\textwidth]{NKU.png}\\[1cm]    
    \textsc{\Huge \kaishu{\textbf{南\ \ \ \ \ \ 开\ \ \ \ \ \ 大\ \ \ \ \ \ 学}} }\\[0.9cm]
    \textsc{\huge \kaishu{\textbf{计\ \ 算\ \ 机\ \ 学\ \ 院}}}\\[0.5cm]
    \textsc{\Large \textbf{网络安全技术作业报告}}\\[0.8cm]
    \HRule \\[0.9cm]
    { \LARGE \bfseries 基于MD5算法的文件完整性校验程序}\\[0.4cm]
    \HRule \\[2.0cm]
    \centering
    \textsc{\LARGE \kaishu{朱浩泽\ 1911530}}\\[0.5cm]
    \textsc{\LARGE \kaishu{年级\ :\ 2019级}}\\[0.5cm]
    \textsc{\LARGE \kaishu{专业\ :\ 计算机科学与技术}}\\[0.5cm]
    \textsc{\LARGE \kaishu{班级\ :\ 计算机科学与技术2班}}\\[0.5cm]
    \vfill
    {\Large \today}
    \end{center}
\end{titlepage}
% -------------摘------要--------------
\newpage
\thispagestyle{empty}
% ----------------------------------------------------------------
\tableofcontents
% ----------------------------------------------------------------
\newpage
\watermark{60}{10}{NKU}
\setcounter{page}{1}
\section{实验目的}
MD5 算法是目前最流行的一种信息摘要算法,在数字签名,加密与解密技术,以及文件完整性检测等领域中发挥着巨大的作用。熟悉 MD5 算法对开发网络应用程序,理解网络安全的概念具有十分重要的意义。
\begin{itemize}
    \item 深入理解 MD5 算法的基本原理。
    \item 掌握利用 MD5 算法生成数据摘要的所有计算过程。
    \item 掌握 Linux 系统中检测文件完整性的基本方法。
    \item 熟悉 Linux 系统中文件的基本操作方法。
\end{itemize}

\section{实验目的}
\begin{itemize}
    \item 准确地实现 MD5 算法的完整计算过程。
    \item 对于任意长度的字符串能够生成 128 位 MD5 摘要。
    \item 对于任意大小的文件能够生成 128 位 MD5 摘要。
    \item 通过检查 MD5 摘要的正确性来检验原文件的完整性。
\end{itemize}

\section{实验步骤及实验结果}
\subsection{实验步骤}
\subsubsection{实验环境}
macOS 12.3(基于 unix), C++11, Cmake
\subsubsection{核心代码实现}
\begin{itemize}
    \item \textbf{MD5类定义实现}\\
    在MD5.h头文件中实现MD5类,该类用于实现MD5算法的基本功能,包括MD5的计算过程,以及MD5的校验过程。
    \begin{lstlisting}[language = C++]
class MD5 
{
private:
    DWORD state[4];  //用于表示 4 个初始向量 
    DWORD count[2];  //用于计数,count[0]表示低位,count[1]表示高位
    BYTE buffer_block[64];  //用于保存计算过程中按块划分后剩下的比特流
    BYTE digest[16];  //用于保存 128 比特长度的摘要 
    bool is_finished;  //用于标志摘要计算过程是否结束 
    static const BYTE padding[64];  //用于保存消息后面填充的数据块 
    static const char hex[16]; //用于保存 16 进制的字符 
    void Stop(); 
    void Transform(const BYTE block[64]);
    //将双字流转换为字节流 
    void Encode(const DWORD *input, BYTE *output, size_t length); 
    //将字节流转换为双字流 
    void Decode(const BYTE *input, DWORD *output, size_t length); 
    //将字节流按照十六进制字符串形式输出 
    std::string BytesToHexString(const BYTE *input, size_t length); 

public:
    MD5(); 
    MD5(const std::string &str); 
    MD5(std::ifstream &in); 
    void Update(std::ifstream &in);
    void Update(const BYTE* input,size_t length);
    const BYTE* GetDigest(); 
    std::string Tostring(); 
    void Reset(); 
};
    \end{lstlisting}
    \textbf{对于私有变量}
    \begin{itemize}
        \item 数组 state 表示四个初始向量
        \item 数组 count 是计数器,记录已经运算的比特数
        \item buffer\_block 是 64 字节的缓存快,保存消息被划分后不足 64 字节的数据,或者保存每次运算的 64 字节数据
        \item digest 用于保存生成的 MD5 摘要
        \item padding 存储填充块
        \item hex 存储 16 个 16 进制字符
    \end{itemize}
    \textbf{对于成员函数}
    \begin{itemize}
        \item Update(const BYTE* input, size\_t length) 对给定字符串进行 MD5 运算
        \item Update 对给定长度的字节流进行MD5运算
        \item get\_digest 获取摘要
        \item Reset 重置初始变量
        \item Transform 对一个 512 比特的消息分组进行 MD5 运算
        \item Decode 将 64byte 的数据块划分为 16 个 32bit 大小的子分组
        \item Tostring 生成 MD5 摘要字符串
    \end{itemize}
    \item \textbf{宏定义}\\
    用宏定义定义四轮计算中的 FF、GG、HH、II 函数,其中 a、b、c、d 表示计算向量,x 表示一个 32 位的子块,s 表示循环左移的位数,ac 表示弧度。于此同时,定义在 MD5 四轮迭代计算中向量 A、B、C、D、循环左移的位数。
    \begin{lstlisting}[language = C++]
typedef unsigned char BYTE;
typedef unsigned long DWORD;

#define BUFFER_SIZE 8

#define S11 7
#define S12 12
#define S13 17
#define S14 22
#define S21 5
#define S22 9
#define S23 14
#define S24 20
#define S31 4
#define S32 11
#define S33 16
#define S34 23
#define S41 6
#define S42 10
#define S43 15
#define S44 21

#define F(x, y, z) (((x) & (y)) | ((~x) & (z))) 
#define G(x, y, z) (((x) & (z)) | ((y) & (~z))) 
#define H(x, y, z) ((x) ^ (y) ^ (z)) 
#define I(x, y, z) ((y) ^ ((x) | (~z))) 

#define ROTATE_LEFT(x, n) (((x) << (n)) | ((x) >> (32-(n))))
#define FF(a, b, c, d, x, s, ac) {(a) += F ((b), (c), (d)) + (x) + ac;(a) = ROTATE_LEFT ((a), (s));(a) += (b);}
#define GG(a, b, c, d, x, s, ac) {(a) += G ((b), (c), (d)) + (x) + ac;(a) = ROTATE_LEFT ((a), (s));(a) += (b);}
#define HH(a, b, c, d, x, s, ac) {(a) += H ((b), (c), (d)) + (x) + ac;(a) = ROTATE_LEFT ((a), (s));(a) += (b);}
#define II(a, b, c, d, x, s, ac) {(a) += I ((b), (c), (d)) + (x) + ac;(a) = ROTATE_LEFT ((a), (s));(a) += (b);}
    \end{lstlisting}
    \item \textbf{Update函数}\\
    将输入 MD5 哈希的字符串进行分块,把输入 按 64 字节分为 N+1 个组,最后一个组不足 64 字节(可能为 0 字节),利用 count 数组和 buffer\_block 数组记录剩余未加密的部分,然后每次加密时核对遗留未加密的数据块,填充后满足一个分组后,传入到 Transform 内核加密函数,得到的结果被存在 state 状态数组中,满足下次加密的时候自动在已经加密的部分继续操作。
    \begin{lstlisting}[language = C++]
void MD5::Update(const BYTE* input,size_t length)
{
    DWORD i, index, partLen;
    is_finished = false;
    index = (DWORD)((count[0] >> 3) & 0x3f);
    if((count[0] += ((DWORD)length << 3)) < ((DWORD)length << 3)) {
        count[1]++;
	}
    count[1] += ((DWORD)length >> 29);
    partLen = 64 - index;
    if(length >= partLen){
        memcpy(&buffer_block[index], input, partLen);
        Transform(buffer_block);
        for (i = partLen; i + 63 < length; i += 64) {
            Transform(&input[i]);
		}
        index = 0;
    } 
	else {
        i = 0;
    }
    memcpy(&buffer_block[index], &input[i], length-i);
}
    \end{lstlisting}
    我们也可以对Update函数进行重载,让其可以直接直接对文件流进行处理。
    \begin{lstlisting}[language = C++]
void MD5::Update(std::ifstream &in) {
	if (!in)
		return;
	std::streamsize length;
	char buffer[BUFFER_SIZE];
	while (!in.eof()) {
		in.read(buffer, BUFFER_SIZE);
		length = in.gcount();
		if (length > 0)
			Update((const BYTE*)buffer, length);
	}
	in.close();
}
    \end{lstlisting}
    \item \textbf{Transform 函数}\\ 
    循环变换是整个MD5算法最核心,也是最复杂的部分。一个512位分组的数据被进一步划分为16个32位的子分组,对每个子分组进行下图所示的变换:
    \begin{center}
        \includegraphics*[scale = 0.2]{v2-66f0ecf6a1e2cde4e2db1908f054f5ae_1440w}
    \end{center}
    图中的F函数代表一次由位运算构成的非线性变换,每一轮循环变换用到的F函数不一样。加号表示加法运算。常数AC的值在每一次变换中都不一样,表达式为$AC_i = int(4294967296|sin(i)|)$ ,i表示第i次变换。左移位数S有规律地周期性变化。数据的16个子分组都参与到上图所示的变换,顺序不定。当16个子分组处理完成时,我们就说完成了一轮循环变换。MD5的一个数据分组一共需要进行四轮的循环变换。将四轮循环变换后得到的A、B、C、D的值分别和原来的值相加,就是A、B、C、D进行循环变换后的结果。

    该函数函数中进行了一系列 MD5 算法的内核运算。初始设置一个初始状态,然后对于这四个向量的初始向量进行四轮函数混淆加密,然后每一轮内部又进行很多函数操作。首先将初始向量 state 的数值赋给变量 a、 b、 c、 d 中。 调用 Decode 函数,将 64 字节的数据块划分为 16 个 32 比特大小的子分组。因为每一轮计算都是对 32 比特子分组进行操作,所以重新划分后可以方便后面的计算过程。依次调用函数 FF、 GG、 HH、 II 展开 4 轮计算,其中每一轮计算包含 16 小步,每一步对一个 32 比特子分组进行运算。函数 FF、 GG、 HH、 II 的前 4个参数是变量 a、 b、 c、d 的不同排列,参数 X[k]表示对第 k 个子分组进行计算, Sij 表示第 i 轮第 j 步计算循环左移的位数。
    
    \begin{lstlisting}[language = C++]
void MD5::Transform(const BYTE block[64]) {
    DWORD a = state[0], b = state[1], c = state[2], d = state[3], x[16];
    Decode(block, x, 64);
    /* 第 1 轮 */
    FF (a, b, c, d, x[ 0], S11, 0xd76aa478); /* 1 */
	FF (d, a, b, c, x[ 1], S12, 0xe8c7b756); /* 2 */
	FF (c, d, a, b, x[ 2], S13, 0x242070db); /* 3 */
	FF (b, c, d, a, x[ 3], S14, 0xc1bdceee); /* 4 */
	FF (a, b, c, d, x[ 4], S11, 0xf57c0faf); /* 5 */
	FF (d, a, b, c, x[ 5], S12, 0x4787c62a); /* 6 */
	FF (c, d, a, b, x[ 6], S13, 0xa8304613); /* 7 */
	FF (b, c, d, a, x[ 7], S14, 0xfd469501); /* 8 */
	FF (a, b, c, d, x[ 8], S11, 0x698098d8); /* 9 */
	FF (d, a, b, c, x[ 9], S12, 0x8b44f7af); /* 10 */
	FF (c, d, a, b, x[10], S13, 0xffff5bb1); /* 11 */
	FF (b, c, d, a, x[11], S14, 0x895cd7be); /* 12 */
	FF (a, b, c, d, x[12], S11, 0x6b901122); /* 13 */
	FF (d, a, b, c, x[13], S12, 0xfd987193); /* 14 */
	FF (c, d, a, b, x[14], S13, 0xa679438e); /* 15 */
	FF (b, c, d, a, x[15], S14, 0x49b40821); /* 16 */

	/* 第 2 轮 */
	GG (a, b, c, d, x[ 1], S21, 0xf61e2562); /* 17 */
	GG (d, a, b, c, x[ 6], S22, 0xc040b340); /* 18 */
	GG (c, d, a, b, x[11], S23, 0x265e5a51); /* 19 */
	GG (b, c, d, a, x[ 0], S24, 0xe9b6c7aa); /* 20 */
	GG (a, b, c, d, x[ 5], S21, 0xd62f105d); /* 21 */
	GG (d, a, b, c, x[10], S22,  0x2441453); /* 22 */
	GG (c, d, a, b, x[15], S23, 0xd8a1e681); /* 23 */
	GG (b, c, d, a, x[ 4], S24, 0xe7d3fbc8); /* 24 */
	GG (a, b, c, d, x[ 9], S21, 0x21e1cde6); /* 25 */
	GG (d, a, b, c, x[14], S22, 0xc33707d6); /* 26 */
	GG (c, d, a, b, x[ 3], S23, 0xf4d50d87); /* 27 */
	GG (b, c, d, a, x[ 8], S24, 0x455a14ed); /* 28 */
	GG (a, b, c, d, x[13], S21, 0xa9e3e905); /* 29 */
	GG (d, a, b, c, x[ 2], S22, 0xfcefa3f8); /* 30 */
	GG (c, d, a, b, x[ 7], S23, 0x676f02d9); /* 31 */
	GG (b, c, d, a, x[12], S24, 0x8d2a4c8a); /* 32 */

	/* 第 3 轮 */
	HH (a, b, c, d, x[ 5], S31, 0xfffa3942); /* 33 */
	HH (d, a, b, c, x[ 8], S32, 0x8771f681); /* 34 */
	HH (c, d, a, b, x[11], S33, 0x6d9d6122); /* 35 */
	HH (b, c, d, a, x[14], S34, 0xfde5380c); /* 36 */
	HH (a, b, c, d, x[ 1], S31, 0xa4beea44); /* 37 */
	HH (d, a, b, c, x[ 4], S32, 0x4bdecfa9); /* 38 */
	HH (c, d, a, b, x[ 7], S33, 0xf6bb4b60); /* 39 */
	HH (b, c, d, a, x[10], S34, 0xbebfbc70); /* 40 */
	HH (a, b, c, d, x[13], S31, 0x289b7ec6); /* 41 */
	HH (d, a, b, c, x[ 0], S32, 0xeaa127fa); /* 42 */
	HH (c, d, a, b, x[ 3], S33, 0xd4ef3085); /* 43 */
	HH (b, c, d, a, x[ 6], S34,  0x4881d05); /* 44 */
	HH (a, b, c, d, x[ 9], S31, 0xd9d4d039); /* 45 */
	HH (d, a, b, c, x[12], S32, 0xe6db99e5); /* 46 */
	HH (c, d, a, b, x[15], S33, 0x1fa27cf8); /* 47 */

	/* 第 4 轮 */
	II (a, b, c, d, x[ 0], S41, 0xf4292244); /* 49 */
	II (d, a, b, c, x[ 7], S42, 0x432aff97); /* 50 */
	II (c, d, a, b, x[14], S43, 0xab9423a7); /* 51 */
	II (b, c, d, a, x[ 5], S44, 0xfc93a039); /* 52 */
	II (a, b, c, d, x[12], S41, 0x655b59c3); /* 53 */
	II (d, a, b, c, x[ 3], S42, 0x8f0ccc92); /* 54 */
	II (c, d, a, b, x[10], S43, 0xffeff47d); /* 55 */
	II (b, c, d, a, x[ 1], S44, 0x85845dd1); /* 56 */
	II (a, b, c, d, x[ 8], S41, 0x6fa87e4f); /* 57 */
	II (d, a, b, c, x[15], S42, 0xfe2ce6e0); /* 58 */
	II (c, d, a, b, x[ 6], S43, 0xa3014314); /* 59 */
	II (b, c, d, a, x[13], S44, 0x4e0811a1); /* 60 */
	II (a, b, c, d, x[ 4], S41, 0xf7537e82); /* 61 */
	II (d, a, b, c, x[11], S42, 0xbd3af235); /* 62 */
	II (c, d, a, b, x[ 2], S43, 0x2ad7d2bb); /* 63 */
	II (b, c, d, a, x[ 9], S44, 0xeb86d391); /* 64 */
    state[0] += a;
    state[1] += b;
    state[2] += c;
    state[3] += d;
}
    \end{lstlisting}
    \item \textbf{Reset函数}\\
    初始化函数,将记忆元件置零,初始化向量进行置位。
    \begin{lstlisting}[language = C++]
void MD5::Reset() {
	is_finished = false;
	count[0] = count[1] = 0;
	state[0] = 0x67452301;
	state[1] = 0xefcdab89;
	state[2] = 0x98badcfe;
	state[3] = 0x10325476;
}
    \end{lstlisting}
    \item \textbf{Stop 函数}\\ 
    用于终止摘要计算过程,输出摘要。具体来说是进行补充尾部以及进行最后的运 算。首先是计算信息总长度和最后一个分组长度,进而计算出需要补足的长度。 第一次调用 Update 函数是补充需要补充的长度并进行一轮运算。第二次 调用 Update 函数是补充信息长度到末尾并进行最后一次运算。
    \begin{lstlisting}[language = C++]
void MD5::Stop() {
	BYTE bits[8];
	DWORD oldState[4];
	DWORD oldCount[2];
	DWORD index, padLen;
	memcpy(oldState, state, 16);
	memcpy(oldCount, count, 8);
	Encode(count, bits, 8);
	index = (DWORD)((count[0] >> 3) & 0x3f);
	padLen = (index < 56) ? (56 - index) : (120 - index);
	Update(padding, padLen);
	Update(bits, 8);
	Encode(state, digest, 16);
	memcpy(state, oldState, 16);
	memcpy(count, oldCount, 8);
}
    \end{lstlisting}
    \item \textbf{Tostring 函数}\\
    将摘要转换成字符串
    \begin{lstlisting}[language = C++]
std::string MD5::Tostring() {
	return BytesToHexString(GetDigest(), 16);
}

std::string MD5::BytesToHexString(const BYTE *input, size_t length) {
	std::string str;
	str.reserve(length << 1);
	for(size_t i = 0; i < length; i++) {
		int t = input[i];
		int a = t / 16;
		int b = t % 16;
		str.append(1, hex[a]);
		str.append(1, hex[b]);
	}
	return str;
}
    \end{lstlisting}


    \item \textbf{Decode 函数}\\
    将 64 字节(64*8 bit)的数据块划分为 16 个 32bit 大小的子分组
    \begin{lstlisting}[language = C++]
void MD5::Decode(const BYTE *input, DWORD *output, size_t length) {
    for(size_t i=0, j=0; j<length; i++, j+=4) {	
        output[i] = ((DWORD)input[j]) | (((DWORD)input[j+1]) << 8) |
            (((DWORD)input[j+2]) << 16) | (((DWORD)input[j+3]) << 24);
    }
}
    \end{lstlisting}

    \item \textbf{Encode 函数}\\ 
    得到最终的结果
    \begin{lstlisting}[language = C++]
void MD5::Encode(const DWORD *input, BYTE *output, size_t length) {
	for(size_t i=0, j=0; j<length; i++, j+=4) {
		output[j]= (BYTE)(input[i] & 0xff);
		output[j+1] = (BYTE)((input[i] >> 8) & 0xff);
		output[j+2] = (BYTE)((input[i] >> 16) & 0xff);
		output[j+3] = (BYTE)((input[i] >> 24) & 0xff);
	}
}
    \end{lstlisting}
\end{itemize}
\subsubsection{控制台函数实现}
设计的 MD5 工具包括以下功能:计算字符串的 MD5 值、计算文件的 MD5 值、文件的 MD5 值校验。

通过命令行参数的形式使用该工具,命令可选参数主要包括:-h、-t、-c、-v、 -f、-s。

\begin{itemize}
    \item -h 是用来显示帮助信息,显示出工具的所有参数以及其基本的使用格式。 只需要简单地进行打印即可,我们在main.cpp文件中利用print\_h函数实现
    \begin{lstlisting}[language = C++]
void print_h(int argc, char *argv[]) {
    if (2 != argc) {
        std::cout << "参数错误." << std::endl;
        return;
    }
    std::cout << "MD5:usage:\n" << "\t" << "[-h] --help information " << std::endl;
    std::cout << "\t" << "[-t] --test MD5 application" << std::endl;
    std::cout << "\t" << "[-c] [file path of the file computed]" << std::endl;
    std::cout << "\t" << "\t" <<"--compute MD5 of the given file" << std::endl;
    std::cout << "\t" << "[-v] [file path of the file validated]" << std::endl;
    std::cout << "\t" << "\t" <<"--validate the integrality of a given file by manual input MD5 value" << std::endl;
    std::cout << "\t" << "[-f] [file path of the file validated] [file path of the .md5 file]" << std::endl;
    std::cout << "\t" << "\t" <<"--validate the integrality of a given file by read MD5 value from .md5 file" << std::endl;
} 
    \end{lstlisting}]
    \item -t 是用来测试 MD5 工具的功能,可以通过该参数来测试 MD5 工具的功能。我们在main.cpp文件中利用print\_t函数实现
    \begin{lstlisting}[language = C++]
void print_t(int argc, char *argv[]) {
    if (2 != argc) {
        std::cout << "参数错误." << std::endl;
        return;
    }
    std::string test[] = {"", "a", "abc", "message digest", "abcdefghijklmnopqrstuvwxyz", 
    "ABCDEFGHIJKLMNOPQRSTUVWXYZabcdefghijklmnopqrstuvwxyz0123456789", 
    "12345678901234567890123456789012345678901234567890123456789012345678901234567890"
    };
    MD5 md5;
    for (int i = 0; i < 7; ++i) { 
        md5.Update((const BYTE*)test[i].c_str(), test[i].length());
        std::cout << "MD5(\"" + test[i] + "\") = " << md5.Tostring()<< std::endl;
    }
}
    \end{lstlisting}
    \item -c 是用来计算文件摘要并打印的。在 -c 参数后面写入文件的路径,首先读入文件,然后利用 MD5 类进行 MD5 值的计算,即可获得此文件的 MD5 值,计算的结果将会显示在终端屏幕上。我们在main.cpp文件中利用print\_c函数实现
    \begin{lstlisting}[language = C++]
void print_c(int argc, char *argv[]) {
    if (3 != argc) {
        std::cout << "参数错误." << std::endl;
        return;
    }
    std::string filePath = argv[2];
    std::ifstream fileStream(filePath);
    MD5 md5;
    md5.Update(fileStream);
    std::cout << "The MD5 value of file(\"" << filePath << "\") is " << md5.Tostring() << std::endl;
}
    \end{lstlisting}
    \item -v 参数用于比对文件的 MD5 哈希值。首先手动读入待比对的 MD5 值,并打印在终端上。然后读取待比对文件,利用 MD5 类进行 MD5 值的计算,将两个 MD5 进行比较后,将对比的结果输出在终端上。通过手动输入待比对的 MD5 哈希值,从而与文件计算出的 MD5 哈希值进行比较,得出文件是否出现更改, 实现文件完整性校验。
    
    首先比较参数argv[1],判断是否通过手工输入进行验证。若是,则继续下面步骤;否则,退出。检测被测文件的路径是否存在,若存在,则继续下面步骤;否则,退出。输入被测文件的MD5摘要并保存在数组InputMD5中。打开被测文件,读取被测文件内容,并调用Update函数重新计算被测文件的MD5摘要。调用Tostring函数将MD5摘要表示成16进制字符串形式。最后调用strcmp函数判断两个摘要是否相同,若相同,则说明被测文件是完整的;否则,说明文件受到了破坏。
    \begin{lstlisting}[language = C++]
void print_v(int argc,char *argv[]) {
    if (3 != argc) {
        std::cout << "参数错误." << std::endl;
        return;
    }
    std::string filePath = argv[2];
    std::cout << "Please input the MD5 value of file(\"" << filePath << "\")..." << std::endl;
    std::string inputMD5;
    std::cin >> inputMD5;
    std::cout << "The old MD5 value of file(\"" << filePath << "\") you have input is" << std::endl << inputMD5 << std::endl;
    std::ifstream fileStream(filePath);
    MD5 md5;
    md5.Update(fileStream);
    std::string genMD5 = md5.Tostring();
    std::cout << "The new MD5 value of file(\"" << filePath << "\") that has computed is" << std::endl << genMD5 << std::endl;
    if (!genMD5.compare(inputMD5)) {
        std::cout << "OK! The file is integrated" << std::endl;
    }
    else {
        std::cout << "Match Error! The file has been modified!" << std::endl;
    }
}
    \end{lstlisting}
    \item -f 输入文件和 MD5 文件,前者计算出摘要与后者比较。程序读取.md5 摘要,重新计算被测文件的 MD5,最后将两者逐位比较。首先比较参数argv[1],判断是否通过.md5文件进行验证。若是,则继续下面步骤; 否则,退出。检测被测文件的路径和.md5 文件的路径是否存在,若存在,则继续下面步骤;否 则,退出。打开.md5文件,读取文件中的记录,调用strtok函数获得被测文件的MD5摘要。打开被测文件,读取被测文件内容,并调用Update函数重新计算被测文件的MD5摘要。调用Tostring函数将MD5摘要表示成16进制字符串形式。最后调用strcmp函数判断两个摘要是否相同,若相同,则说明被测文件是完整的;否则,说明文件受到了破坏。
    \begin{lstlisting}[language = C++]
void print_f(int argc, char *argv[]) {
    if (4 != argc) {
        std::cout << "参数错误." << std::endl;
        return;
    }
    std::string filePath = argv[2];
    std::string md5Path = argv[3];
    std::ifstream md5Stream(md5Path);
    std::string oldMD5Str((std::istreambuf_iterator<char>(md5Stream)), std::istreambuf_iterator<char>());
    oldMD5Str = (std::string)strtok(const_cast<char*>(oldMD5Str.c_str())," ");
    std::cout << "The old MD5 value of file(\"" << filePath << "\") in " << md5Path << " is " << std::endl << oldMD5Str << std::endl;
    std::ifstream fileStream(filePath);
    MD5 md5;
    md5.Update(fileStream);
    std::string genMD5 = md5.Tostring();
    std::cout << "The new MD5 value of file(\"" << filePath << "\") that has computed is" << std::endl << genMD5 << std::endl;
    if (!genMD5.compare(oldMD5Str)) {
        std::cout << "OK! The file is integrated" << std::endl;
    }
    else {
        std::cout << "Match Error! The file has been modified!" << std::endl;
    }
}
    \end{lstlisting}
    \item 主函数
    \begin{lstlisting}[language = C++]
int main(int argc,char *argv[]) { 
    std::unordered_map<std::string, void(*)(int, char*[])> mapOp = {{"-t", print_t}, {"-h", print_h}, {"-c", print_c}, {"-v", print_v}, {"-f", print_f}};
    if (argc < 2) { 
        std::cout << "参数错误,argc = " << argc << std::endl;
        return -1;
    }
    std::string op = argv[1];
    if (mapOp.find(op) != mapOp.end()) {
        mapOp[op](argc, argv);
    }
    return 0;
}
    \end{lstlisting}
\end{itemize}




\subsection{实验结果}
\begin{center}
    打印帮助信息
    \includegraphics*[scale = 0.3]{1}

    打印测试信息
    \includegraphics*[scale = 0.3]{2}

    为指定文件生成MD5摘要
    \includegraphics*[scale = 0.3]{3}

    验证文件完整性方法一
    \includegraphics*[scale = 0.3]{4}
    \includegraphics*[scale = 0.3]{5}

    验证文件完整性方法二
    \includegraphics*[scale = 0.3]{6}
\end{center}
\section{实验遇到的问题及解决方法}
本次实验指导书对算法描述的比较详细,代码也给出了一定的框架,所以实现起来并没有什么结构性的大的问题,只有一些小的细节没有注意到。
\begin{enumerate}
    \item 我们在使用unordered\_map数据结构存储string类型和函数类型的变量时,出现了以下报错
    \begin{center}
        \includegraphics*[scale = 0.2]{error}
    \end{center}
    经过查阅资料,我们发现,该类型的stl容器要想匹配该功能,需要C++11以及以上的版本才可以实现,所以我们在CmakeLists中添加 \lstinline{set (CMAKE_CXX_STANDARD 11)},指定C++版本为C++11后,重新生成makefile,便解决了问题。

    \item 我们在MD5算法类完成的时候,进行了测试,发现了如下报错
    \begin{center}
        \includegraphics*[scale = 0.3]{error1}
    \end{center}
    根据报错内容,我们可以看到,报错信息是我们没有在MD5类中声明hex和padding。但是我们仔细检查了MD5类的源代码,类中确实定义了hex和padding变量。
    \begin{center}
        \includegraphics*[scale = 0.5]{z z}
    \end{center}
    经过反复查阅源代码和实验手册后,我们发现我们没有对这两个变量进行赋初值的操作,这时我们意识到,static变量需要在cpp文件或者头文件里初始化,即使其是类的变量。

\end{enumerate}
\section{实验结论}
本次实验相对来说较为简单,指导书上提供的框架也较为齐全,我们只要在原有的框架上进行一些改动即可。通过手动实现MD5算法,我们了解了其设计细节,并利用C++
高效的实现了该算法,提高了我们对信息安全领域的认知和兴趣。


\end{document}